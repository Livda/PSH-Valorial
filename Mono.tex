\documentclass[a4paper,10pt]{report}
\usepackage[utf8x]{inputenc}
\usepackage[T1]{fontenc}
\usepackage[french]{babel}
\usepackage{graphicx,eso-pic,eurosym,tabularx}
\usepackage{cite}
\usepackage{hyperref}
\hypersetup{
    colorlinks=true,                         
    citecolor=black, % Couleur des numéros de la biblio dans le corps
    urlcolor=blue  } % Couleur des url
\usepackage{}
\newcommand\BackgroundPic{%
	\put(0,250){%
		\parbox[b][\paperheight]{\paperwidth}{%
			\vfill
			\centering
			\includegraphics[width=\paperwidth,height=\paperheight,keepaspectratio]{logoINSA.jpg}
			\vfill
}}}


\title{La place de Valorial dans l'économie bretonne}
\author{ \bsc{Aurélien Fontaine}
	\and \bsc{Manuteau Huang} 
	\and \bsc{Nicolas Le Borgne}
	\and \bsc{Maxime Cadoret}
	\and \bsc{Flavien Lecuyer}
}
%\institute[INSA de Rennes]{Institut National des Sciences Appliquées de Rennes}
\date{\today}



\begin{document}
	\AddToShipoutPicture*{\BackgroundPic}
	\maketitle
	\setcounter{tocdepth}{2}
	\tableofcontents
	
\chapter*{Introduction}
\addcontentsline{toc}{chapter}{Introduction}
	 L’industrie agro-alimentaire tient depuis longtemps une place prépondérante dans l’économie bretonne. En effet, celle-ci représente plus de 40 du secteur industriel régional ainsi que 15\% de la valeur ajoutée créée pour ce secteur au niveau national. 
	 
	 L’importance de l’agro-alimentaire dans le tissu économique breton rend d’ailleurs de la région une image très rurale, laquelle tend à être préservée tant l’évolution de ce secteur est significative. Malgré la baisse conséquente du nombre de salariés dans le secteur manufacturier entre 2010 et 2012, qui ont ainsi été réduit de 42\%, l’agro-alimentaire reste la principale source d’emploi dans la région. La part de l’agro-alimentaire dans la masse salariale de l’industrie manufacturière est passée à près de deux emplois sur trois (37 à 62\%).
	 
	 Malgré tout, l’économie agro-alimentaire ne garde qu’une importance faible dans l’apport de richesses au territoire local. La prédominance des industries de première transformation provoque ainsi une baisse des taux de valeur ajoutée par rapport à la moyenne nationale des IAA (Industries Agro-Alimentaires). Cela fragilise ainsi cette économie, qui doit donc réduire les prix afin de rester compétitive face aux marchés étrangers tout en luttant contre les lois françaises de plus en plus strictes. A titre d’exemple, les années 2012 et 2013 ont été marquées par des restructurations importantes touchant essentiellement le secteur de l’abattage et de la première transformation. Ces restructurations, qui pourraient se poursuivre, ont eu des conséquences sur l’emploi dans des territoires fragilisés, notamment dans le Morbihan et dans le Finistère. Les difficultés subies dans l’agro-alimentaire, en particulier pour les filières porcines et volaillère, ont créé des réactions en chaîne, qui ont de ce fait dépassé le cadre de l’agroalimentaire.
	 
	 Dans le but de mieux répondre à ces problèmes, on constate de la part des entreprises une volonté forte de se diversifier au travers de l’innovation. En l’espace de dix ans, les industriels ont ainsi réussi à mettre en place de nouvelles gammes de produit, permettant donc d’augmenter leur taux de valeur ajoutée et par extension placer la Bretagne à la première place en termes de valeur ajoutée dans le classement français. Pourtant, l'innovation reste un phénomène trop peu fréquent. On remarque ainsi un manque cruel d'entreprises innovantes dans le secteur.
	 
	 Cela donne par conséquent une place spécifique aux problèmes de l'innovation dans l’économie bretonne. Nous pouvons ainsi nous demander « Quelle est la place de l'innovation de l'industri agro-alimentaire bretonne et commen peu-elle évoluer ? » Afin de répondre à cette question, nous nous intéresserons de prime abord à la situation de l’agro-alimentaire dans la région Bretagne. Dans un second temps, nous présenterons les avantages que tirent les entreprises de l'innovation. Pour finir, nous nous pencherons sur les problèmes que cela pose et sur les nouvelles perspectives d'innovation.
	 
\chapter{La situation agro-alimentaire bretonne}
	Aujourd’hui, et depuis quelques années, la situation de l’industrie agro-alimentaire tend à se dégrader à l’international. Pour autant, la Bretagne semble bien s’en sortir, comme en attestent les chiffres. Cela rend donc la situation de cette industrie particulière. Afin de la présenter en détail, nous commencerons par voir la situation de l’agro-alimentaire français à l’international. Ensuite, nous comparerons la situation de la Bretagne à celle de la France. Pour finir, nous parlerons de la crise dans l’agro-alimentaire et des actions menées pour la gérer.

	\section{L’agro-alimentaire français dans le marché international}
		Comme beaucoup le savent, l’agro-alimentaire est dans une situation peu avantageuse depuis quelques années, affaiblie par plusieurs crises en quelques années. Afin de donner un état des lieus de cette industrie, nous verrons tout d’abord les problèmes qu’ont les entreprises à conserver leur compétitivité. Dans un second temps, nous nous intéresserons aux problèmes de l’emploi dans le secteur. Pour finir, nous verrons que ce mauvais bilan se rattrape au travers de la technologie et de l’export.

		\subsection{La compétitivité en baisse}
			Pour commencer, on remarque que la compétitivité de l’industrie agro-alimentaire est dans une tendance à la baisse.
			
			Au cours des années 2000, la France était le premier exportateur agro-alimentaire européen. Aujourd’hui l’Allemagne et les Pays-Bas sont passés devant la France qui se place en troisième position, et il arrive que certains secteurs de l’industrie agro-alimentaire comme celui de la viande ont une balance commerciale déficitaire : “le déficit commercial s'est creusé de 20\% au 2e trimestre 2013 sur un an”.
			Ce recul de la France est dû, entre autres, aux nombreux obstacles qui “freinent le développement de cette industrie”. Ils sont d’ordres multiples. Par exemple, il y a une certaine difficulté à recruter de nouveaux employés, plus particulièrement des personnes qualifiées. Le secteur de l’agro-alimentaire n’est pas très attirant suite à la croissance du monde numérique mais nécessite une grande main d’oeuvre.
			De plus, les investisseurs comme les banques ou les grands groupes sont de moins en moins enclins à accorder des fonds aux entreprises. Or le coût des matières premières ne cesse de croître alors que les prix pour le client final stagnent. Les entreprises ont donc une marge de manoeuvre de plus en plus réduite pour développer de nouvelles méthodes. De surcroît, la législation française est très contraignante.

		\subsection{Le paradoxe de la productivité et de l'emploi}
			Pour ce qui est de l’emploi, on constate aujourd’hui un paradoxe important. En effet, les industriels de l’industrie agro-alimentaire sont de plus en plus productifs, mais on se heurte en parallèle avec une baisse de plus en plus conséquente du nombre d’emplois et d’organisations dans le secteur.
			
			Certains moyens ont été mis en place afin d’aider les entreprises comme des sites de recrutement sur internet : www.jobagroalimentaire.com ou www.apecita.com .  “Ces mauvais indices surviennent dans un contexte de baisse de la production de 2,6\% du secteur [de la viande].
			Ce climat pèse bien sûr sur l'emploi puisque la filière a perdu 7.200 emplois entre juin 2013 et juin 2012”

		\subsection{L'export et la technologie de pointe, des activités qui portent}
			Malgré ce bilan plutôt négatif en ce qui concerne l’agro-alimentaire français, il reste à souligner les points positifs dans l’évolution de ces dernières années. En effet, les entreprises qui se tournent vers l’export et la technologie de pointe arrivent à garder une place importante sur le marché, voire même à se développer.

	\section{La Bretagne : un cas particulier}
	Nous avons donc vu que la situation de l’industrie agro-alimentaire en France est de moins en moins bonne. Pour autant, elle semble avoir un bon ancrage dans l’économie bretonne. De manière à expliquer les caractéristiques de ce secteur en Bretagne, nous étudierons pour débuter la place de la Bretagne en France, aussi bien géographiquement qu’économiquement. Deuxièmement, nous parlerons de la place importante de l’industrie agro-alimentaire en Bretagne. Pour terminer, nous étudierons les secteurs d’activités de l’agro-alimentaire en Bretagne.

		\subsection{La place de la Bretagne en France}
			L’une des raisons pour lesquelles l’industrie agro-alimentaire se porte mieux en Bretagne qu’en France s’explique par la position de la région, aussi bien économiquement que géographiquement. En effet, la région présente de multiples spécificités qui font d’elle un cas à part.
			
			En effet, la Bretagne possède depuis longtemps un tissu rural très présent, de la tradition des goémoniers aux éleveurs de porcs, la Bretagne a, pour le reste de la France, une image campagnarde indépendante. Mais en plus de son histoire avec l’agriculture, la Bretagne s’est développée dans ce sens grâce à des spécificités géographiques. Nous pouvons voir grâce à ces cartes (figure 1) la répartition de la population dans la région. Le massif armoricain étant une chaîne de montagnes de 65 000 km² culminant à un peu plus de 400m d’altitude, il a pour principal relief de vastes landes escarpées où la seule activité subsistante est l’agriculture et ses vastes pâturages d’élevage.
			
			A cause de ces reliefs, nous pouvons aussi constater que la population de la Bretagne est localisée principalement proche des littoraux ou bien du bassin Rennais (figure 2). Cette répartition a conduit au développement des littoraux bretons au niveau portuaire. Les trois principaux ports de trafic de marchandises étant Saint-Malo, Brest et Lorient qui ont réalisé en 2012 84 \% du trafic total de la région, pour un total de 8.297 millions de tonnes aux trois quarts destinés à l’Union Européenne ou au reste de la France. Ce sont également plus de 3.9 millions de passagers qui ont transité par les ports de la péninsule bretonne au cours de cette année 2012. On peut donc constater un développement important du commerce maritime breton qui, en majeure partie, est occupé par les produits agricoles et alimentaires (35 \%), devant les produits énergétiques (24 \%) puis la métallurgie (18 \%).
			
			Mais même si la Bretagne se tourne vers l’extérieur grâce à ses échanges maritimes et que les deux tiers du trafic total de marchandises sont des échanges intra régionaux à hauteur de 120 millions de tonnes, les échanges nationaux représentent 49 millions de tonnes de marchandises donc 42\% sont des produits agricoles et alimentaires. Ces échanges se font principalement avec les régions limitrophes de la Bretagne (90 \%). En effet, en tant que première région agricole française en termes de production, le niveau d’importation de denrées alimentaires reste limité à celles que le climat régional rend impossible à cultiver. Le fait que la Bretagne soit également la première région française au niveau de la pêche et que l’élevage soit une part importante de la production agricole, rend l’importation de ces produits une pratique très rare.
			
			Au travers de cette présentation de la Bretagne et de la place qu’occupe cette région en France, nous avons pu constater que la tendance historique d’indépendantisme breton était liée à sa capacité à produire la majeure partie de ses produits de consommation en tant que première région agricole de France. Les échanges hors de la région sont principalement des exportations, par exemple en 2006, les exportations de la Bretagne représentaient 8.9 milliards d’euros contre 7.8 milliards d’euros en importation. La Bretagne occupe donc une place importante dans la production agricole française, qui était en 2012 considérée comme la 2e puissance agricole mondiale derrière les Etats-Unis.
			
	
		\subsection{L'importance de l'agro-alimentaire dans l'industrie bretonne}
			Comme expliqué précédemment, la Bretagne est la région Française la plus productive au niveau de l’agro-alimentaire et de la pêche. C’est le secteur de l’industrie manufacturière le plus employeur. Si on regarde les chiffres de 2012 donnés par le ministère de l’agriculture, 58 474 emplois sur 94 173, soit 62.1\% sont dans le secteur de l’IAA.
			
			L’industrie agro-alimentaire n’est pas seulement prépondérante dans l’industrie manufacturière, elle représente à elle seule 40\% des emplois industriels tous secteurs confondus et 10\% des emplois de la région.
			
		\subsection{De multiples secteurs d'activités}
			Enfin, on peut expliquer la situation de l'agro-alimentaire en Bretagne par la multiplicité des secteurs d'activités existants dans la région. En effet, non seulement cette industrie est importante dans l'économie bretonne, mais elle se présente aussi sous plusieurs formes.
			
			
			Si l’on regarde les chiffres de 2012, on constate qu’un peu plus de 46\% des emplois du secteur sont liés aux activités de  “Transformation et conservation de la viande et préparation de produits à base de viande”. Vient ensuite la fabrication d’aliments divers et de produits laitiers. Ces chiffres sont également liés à la taille des entreprises du secteur, en effet la majeure partie des salariés sont embauchés par des établissements de plus d’une centaine de personnes. En 2012, il y avait 4 établissements de plus de 1000 salariés :
			Kermene, qui est le premier fournisseur des magasins de l’enseigne E.LECLERC en produit de boucherie et charcuterie.
			COOPERL ARC ATLANTIQUE, Groupe coopératif agricole majeur de la production porcine, qui est l’un des leaders bretons de la production porcine.
			SOCIETE VITREENE D’ABATTAGE - SVA Jean Rozé qui est spécialisée dans la conservation et la transformation de la viande de boucherie.
			GROUPE BIGARD qui est le leader européen au niveau de la viande bovine ainsi que le n°1 français de la viande (d’après leur site)
			
			Tel que l’on peut le constater, les 4 plus gros établissements du secteur AA breton en 2012 étaient liés à l’élevage, à la transformation et à la conservation de la viande. Ces 4 établissements représentent 10\% des emplois du secteur et le reste des établissements les plus recruteurs sont pour la plupart reliés au secteur de la viande. Ces grosses entreprises confirment bien la tendance de la région car c’est un secteur propice à l’exportation et qui représente 57\% des ventes à l’exportation de l’IAA régionale.
			
			Les autres secteurs sont bien moins importants que celui de la viande mais la filière laitière, qui représente 10\% de l’effectif salarié du secteur de l’IAA, reste tout de meme la première de France au niveau de la production. En effet, tous les grands groupes y sont implantés tels que Lactalis, Bongrain, Danone, Entremont Alliance et bien d’autres. La Bretagne élève ainsi un tiers des vaches laitières françaises.
			
			Parmis les produits laitiers issus de cette activité, le produit phare de la Bretagne reste le beurre en version salée. Mais de nombreuses spécialités culinaires bretonnes telles que les crêpes, le kouign-amann, le sablé breton ou encore le caramel au beurre salé découlent de cette agriculture laitière si prononcée dans la région.
			
			La Bretagne voit également émerger des secteurs innovants dans le domaine de l’AA tels que les applications à base d’algues qui servent à diverses activités\cite{CartesBretagneAgroAlimentaire20142016} : complément alimentaire, consommation directe, cosmétique, phytosanitaire ou bien pour la médecine douce.
			
			Au travers de cette étude des différents secteurs d’activité de l’industrie agro-alimentaire bretonne, bien que non exhaustive, nous pouvons constater un très fort monopole de la filière viande et laitière car ce sont des filières très sujettes à l’exportation. Certains secteurs essayent tout de même de se développer en s’appuyant sur des transformations de produits innovantes. Ceux-ci sont majoritairement portés par la production d’algues malgré le fait que l’importation d’algues étrangères reste tout de même plus compétitif que de les récolter sur place à cause du manque de structure.
			
			
	\section{La crise, impact et réponses}
		Bien que la Bretagne s’en sorte bien dans la crise de l’agro-alimentaire, cette dernière reste importante et pose de nombreux problèmes. Dans l’optique de savoir ce qui se passe autour de cette crise, nous commencerons par comparer ses impacts au niveau national et au niveau breton. Nous verrons ensuite quelles sont les actions menées par les pouvoirs publics. Enfin, nous parlerons des solutions envisagées pour l’avenir.
		
		\subsection{Des conséquences différentes entre la France et la Bretagne}
			\paragraph{}Avant de pouvoir parler de la réaction générale face à la crise, regardons pour commencer ses impacts. Afin de mieux nous rendre compte de la différence des conséquences en France et en Bretagne, nous procéderons tout d'abord à une analyse de l'impact aux différentes échelles, puis nous comparerons les résultats.

			\paragraph{}Dans un premier temps, intéressons-nous à ce que représente la crise au niveau national. En raison de l'importance que l'industrie agro-alimentaire a dans le pays, il paraît normal de penser que le secteur ait subi la crise de plein fouet, au moins sous certaines formes. La première des conséquences de la crise dans l'agro-alimentaire français est une conséquence de centralisation. En effet, les petites organisations résistent moins bien face à de tels évènements, poussant ainsi les plus grosses structures à effectuer des rachats ou à améliorer leur solidité. Par exemple, le groupe Intermarché a effectué le rachat des abattoirs Gad en Octobre 2014 pour les intégrer à sa filiale SVA Jean Rozé. Ces mêmes rachats ont pour but, entre autres, de diversifier leurs activités pour conforter une place qui tend à se fragiliser. En outre, avec les baisses des aides de la Politique Agricole Commune (pour l’exemple, l’Union Européenne a diminué de 24,5\% les subventions européennes et de 45,4\% les subventions nationales), les ressources financières du secteur se raréfient dangereusement, mettant en péril certaines branches de l’industrie. On peut citer, à titre d’exemple, la filière de la volaille, dont plusieurs représentants, à l’instar de Doux se retrouvent au bord de la fermeture. Cela provoque également une augmentation de l’importation, puisque le solde entre production et consommation diminue de plus en plus et de plus en plus vite (-22\% de 1996 à 2000 puis -40\% de 2000 à 2005)\cite{AvenirExploitationVolailleBretonne}.

			Qui plus est, la crise se manifeste aussi par une crise de confiance. Effectivement, outre la “vache folle de 1986 à 1996”, de multiples scandales se sont accumulés dans l’agro-alimentaire sur la dernière décennie\cite{Scandales} : la grippe aviaire en 2006, les problèmes sur les produits importés la même année, le lait à la mélamine en 2008, l’Escherichia Coli dans les concombres en 2011, les steaks hachés contaminés en 2012, la fraude à la viande de cheval en 2013. À cause de cela, les modes de consommations changent. Plus précisément, les consommateurs se tournent vers les commerces de proximité, car les différents scandales sont majoritairement nés de l’import en masse des grands groupes industriels.
			
			D’un autre côté, des spécificités se présentent à l’échelle régionale. En Bretagne, l’agro-alimentaire est une industrie relativement importante pour l’économie, mais c’est surtout une force majeure pour la région, comme il a été dit précédemment. Ainsi, on voit que même si certaines organisations sont en difficulté, la présence de l’agro-alimentaire à différents niveaux permet de se passer du recours à des intermédiaires étrangers. Ce fait met donc en évidence l’existence d’une synergie entre les acteurs de la région.

			De ce fait, nous pouvons mettre en opposition deux situations : celle de la Bretagne et celle de la France. Ainsi, on voit que le secteur se retrouve dans une position difficile au niveau de la France, en raison de l’existence de marchés étrangers plus compétitifs et donc plus avantageux que les marchés locaux pour les distributeurs, mais aussi à cause du manque de confiance croissant des consommateurs. Mais l’état des choses en Bretagne est tout autre, puisque la région reste ancrée dans cette industrie, comme il a été précisé plus tôt, et que le soin que la région y porte lui offre un délai permettant de trouver des réponses plus efficaces et donc de faire plus aisément face à la crise.

			\paragraph{}Ainsi, nous pouvons sans trop de problèmes affirmer que la Bretagne est dans une meilleure situation que la France
			
		\subsection{Différentes actions à différentes échelles}
			\paragraph{}En réponse à cette crise, les pouvoirs publics ont mis en œuvre de nombreuses actions, plus ou moins efficaces. Mais ces actions ne sont pas toutes venues du même échelon. En effet, certaines d'entre elles ont été menées au niveau national, d'autre à des niveaux plus locaux. Afin de mieux nous en rendre compte, nous commencerons par voir les actions du gouvernement. Ensuite nous verrons ce qui a été fait par la région et les départements.

			\paragraph{}Tout d’abord, le gouvernement a apporté des mesures dans le but de répondre aux problèmes de l’industrie agro-alimentaire. Pour commencer, le gouvernement a décidé en décembre 2013 de mettre en place le Plan Agricole et Agroalimentaire pour l’Avenir de la Bretagne au sein du Pacte d’Avenir\cite{PacteAvenirBretagne}. Cette action vise à apporter à la Bretagne des solutions à la fois pour pallier au déficit d’emplois de ces dernières années, mais aussi à soutenir les entreprises en difficulté. Pour ce faire, des subventions sont prévues pour les entreprises en difficulté, ce qui permet à la fois d’éviter les licenciements et d’aider les entreprises à se redresser. En outre, ce programme cherche à accorder plus d’importance au renouvellement des activités, par l’intermédiaire de l’innovation par exemple. De manière justement à mettre l’accent sur ce point, un soin tout particulier a été accordé aux pôles de compétitivité et aux instituts de recherche. Ainsi, l'une des initiatives prises est la mise en place de l’institut Carnot pour aider les entreprises à vérifier l’état des choses dans leur activité. L’importance donnée aux pôles de compétitivité vise quant à elle à rendre les entreprises capables d’améliorer et d’élargir leurs gammes de produits en fonction de ce qui est en vogue.

			Pour répondre aux questions que se posent les consommateurs sur la qualité des produits qu’ils achètent, une autre mesure a été votée le 11 Février 2014 pour promouvoir la viande d’origine française au travers de sept logos (selon le type de viande en question). Cette décision fait suite à la crise de la fraude à la viande de cheval et a pour but de permettre au consommateur de savoir d’où vient ce qu’il mange et d’être ainsi plus sûr de la qualité du produit. Néanmoins, avec le nombre important d’importations pour tout ce qui est animal, l’apparition de ce logo reste encore très rare. En Septembre, un constat a été fait sur les étals. Mis à part pour la volaille où 55\% des produits mis en rayon affichent le logo, les chiffres sont catastrophiques. Seulement un peu plus de 2\% des produits à base de porc affichent ce logo, ce qui prouve le peu d’importance donnée à l’origine de la viande par les entreprises\cite{FlopVDF}.

%IMAGE VIANDE DE FRANCE

			Une autre des mesures prises l’est par la région. En effet, même si le conseil régional a participé à la création du Plan cité précédemment, des décisions se voulant à la fois plus précise et plus en accord avec la situation régionale ont été faites. La politique régionale, depuis le début de l’année 2014, est d’améliorer la qualité des produits proposés par l’industrie agro-alimentaire\cite{FavoriserQualiteAgricultureAgroalimentaire}. Pour cela, l’accent est mis sur le respect de l’environnement et la production d’énergies propres. Nous pouvons constater que cela fait suite aux crises de l’agro-alimentaire qui ont provoqué la perte de confiance des consommateurs. Ainsi, en produisant mieux plutôt que plus, la région compte répondre aux attentes de la société actuelle en matière de consommation.

			Enfin, les entreprises elles-mêmes cherchent à prendre des mesures pour sauver leur activité. Nous pouvons, pour cela, citer une seconde fois le rachat des abattoirs Gad par Intermarché. Cette action se motive non seulement par des choix économiques dans le groupe, mais également par la volonté de fortifier la SVA Jean Rozé dans son activité d’exploitation de viande animale. Ainsi, en effectuant ce que l’on appelle l’intégration verticale, qui est pour une entreprise l’extension de ses activités le long de sa filière, le groupe se détache d’intermédiaires coûteux et présentant des risques pour sa pérennité. En outre, en étendant son activité, la filiale peut se pencher sur la question d’une modification de la gestion de son activité, et se moins se focaliser sur son cœur de métier tout en déplaçant des ressources sur la recherche et le développement de produits innovants.

			\paragraph{}En conclusion, des actions différentes ont été mises en place à plusieurs niveaux, mais il est important de noter que ces actions rejoignent majoritairement le même but : innover pour plaire au consommateur.

			
		\subsection{Quelles solutions pour l'avenir ?}
			Suite aux problèmes qui se posent aujourd'hui dans l’agro-alimentaire, il est important de se rendre compte des solutions envisagées pour l’avenir de cette industrie. Pour voir cela, nous commencerons par voir quelles sont les solutions envisagées par les entreprises. Ensuite, nous nous intéresserons à ce que prévoit l’État pour les prochaines années.
			
			Pour débuter, voyons quelles actions les entreprises considèrent pour les prochaines années. Pour beaucoup d’entreprises, la solution se trouve dans l’intégration verticale. Ainsi, de nombreux groupes étendent leur activité pour avoir un meilleur contrôle sur la filière. Pour autant, cette solution n’est l’apanage que de grandes structures. Par conséquent, les plus modestes entreprises se tournent vers un autre moyen de faire face : l’innovation. On constate aujourd’hui que le nombre d’entreprises dans le secteur de l’agro-alimentaire se répartit de manière peu équitable. En effet, en 2008, plus de 90\% de ces entreprises étaient des PME\cite{IAAFranceChiffres}. Cela montre ainsi que ce n’est pas forcément l’importance de l’organisation qui lui permet de perdurer dans cette industrie, mais plutôt son activité. C’est pour cette raison qu’un nombre sans cesse plus important d’entreprise tente de restructurer son activité afin de moderniser leur image et de renouveler l’intérêt que les consommateurs peuvent leur porter. Un autre constat est la dispersion des entreprise sur un plan géographique. Non seulement beaucoup d’entreprises sont éloignées les unes des autres, mais en plus elles ne cherchent que peu la collaboration. C’est pourquoi les plus importantes organisations veulent créer dans les prochaines années une synergie qui paraît être la clé pour la solution à de nombreux problèmes.
			Ensuite, Le gouvernement prévoit lui aussi des actions. Dans ses prochains plans économiques, il compte continuer sur la même voie afin de solidifier ce qui est d’ores et déjà en place. Ainsi, on peut se rendre compte que le Plan Agricole et Agroalimentaire pour l’Avenir de la Bretagne au sein du Pacte d’Avenir précité prévoit des réponses sur le court voire moyen terme, plus précisément pour l’horizon 2020 au maximum. Mais que prévoient les pouvoirs publics pour après cette date ? Le ministère de l’agriculture a décidé de mettre en avant trois grands points sont au goût du jour pour 2025\cite{AvenirFiliereAgricole2025}. Le premier d’entre eux est la réponse aux aléas. En effet, avec la multiplication des catastrophes naturelles ces dernières années (et en particulier l’année 2014). Un autre de ces aléas est un aléa plus financier : de plus en plus, les prix dans l’agro-alimentaire sont sujets à la volatilité, en raison des spéculations fortes sur ce secteur. Sur un domaine plus proche de l’industrie, l’État compte renforcer la coopération entre les entreprises, au moyen de simplification des démarches pour celles qui le souhaitent. De ce fait, une synergie peut se mettre en place et donc provoquer une amélioration pour chaque acteur. Pour terminer, le dernier point pensé par le ministère est la formalisation de stratégies commerciales pour développer plus efficacement les échanges entre entreprises et la communication entre entreprises et consommateurs. Ces mêmes stratégies commerciales se veulent également moyen de consolider les marchés d’exportation, qui représente l’une des activités qui se portent le mieux pour l’industrie agro-alimentaire.
			
			Pour conclure, on voit qu’il existe des solutions de différents ordres pensés pour faire face à la crise dans les prochaines années. Certaines correspondent à la vision qu’à chaque entreprise de son activité, tandis que d’autres partent de la situation économique actuelle au niveau national.
			
\chapter{L'innovation, un remède contre la crise ...}
	
	\section{Les différentes formes d'innovation}
	
		\subsection{L'innovation sur les modes de production}
			L’agriculture bretonne s’est petit à petit intensifiée, en effet, depuis les années 70, les travaux se sont de plus en plus mécanisés, les machines sont devenues quasiment omniprésentes dans l’agriculture de nos jours. De plus, par le développement de la recherche sur l’agriculture, de nouveaux outils sont apparus pour permettre “l’augmentation de la productivité et une meilleure sécurité alimentaire”\cite{FAOStatisticalYearbook2013}. Parmi ces outils, on peut citer le GPS dans un tracteur moderne\cite{RobotsChamps} qui permet d’optimiser son circuit dans le cadre de répartition d’engrais.
			
			%IMAGE GPS TRACTEUR
			
			L’utilisation de machines, d’engrais et d’arrosoirs automatisés permet un meilleur rendement qu’une pousse naturelle, c’est-à-dire sans intervention de l’homme après ensemencement. On en vient à parler de mécanisation des champs, ce qui facilite le travail pour l’agriculteur, dans un monde où les parcelles s’agrandissent et où la demande de nourriture est de plus en plus importante.
			
			Du côté de l’élevage, on note la recherche sur le bon développement des animaux, sur le développement de l’élevage en batteries et, entre autres, le rationnement en oligo-éléments\cite{OligoElements}. Effectivement, les bovins, porcins et autres animaux d’élevage ont besoin d’un apport en différents minéraux. Cette affirmation en elle-même est le résultat d’études sur le métabolisme de ces animaux, ce qui débouche sur un contrôle de l’alimentation plus précis et une amélioration de la sécurité alimentaire.
			
			Evidemment, toute cette émulation autour de l’agriculture a aussi permis de développer d’autres modes de production peut-être moins éreintant pour le sol et les animaux comme la rotation des cultures qui permet de préserver la fertilité des sols et de diversifier la production.
			
			%IMAGE ROTATION
			
			Ce qui est important de souligner c’est que les innovations ne se répercutent pas instantanément sur les exploitations, nous exposerons plus loin dans la monographie cette inertie.
			
			Nous verrons plus loin les critiques portant sur le développement des nouvelles technologies, mais nous devons garder en mémoire qu’il est nécessaire d’évoluer pour rester compétitif.
			
			
		\subsection{L'innovation sur l'organisation des entreprises}
				
		\subsection{L'innovation par les nouvelles ressources}
			Les algues sont très demandées par le monde de la cosmétique pour leurs vertus. 
			
	\section{Comment fait-on pour innover ?}
			
		\subsection{La recherche au sein des entreprises}
			Un projet d’innovation commence par une idée qui peut apparaître pour différentes raisons. La plus commune est la mise en évidence d’un besoin qui déclenche la recherche d’une solution. Par exemple le plan Ecophyto\cite{RobotsChamps} qui a pour but de limiter l’utilisation de désherbant chimique pourrait provoquer l’apparition de projets de machines de désherbage automatisés. 
			Des découvertes scientifiques peuvent également ouvrir de nouveaux marchés et de nouvelles possibilités pour les entreprises. 
			
			On peut par exemple citer Bellegum\cite{Bellegum} qui a pu, grâce à Valorial, lancer une gamme innovante de mousse aux légumes. Grâce à ces produits, l’entreprise a pu se hisser au niveau mondial, en voyant le groupe Intermarché proposer ses produits en rayon depuis début 2013.
			Enfin les interactions et la coopération avec les différents acteurs permettent aux entreprises de faire des innovations organisationnelles.
			
			Pour démarrer et aller au bout d’un processus de recherche, il ne suffit pas d’avoir une idée mais il faut également trouver les ressources la mener à bien.
			
			L’innovation dans le milieu industriel est avant tout issu de l’apprentissage des entreprises. Cet apprentissage est directement lié à la capacité de l’entreprise à gérer l’expérience qu’elle produit. En effet, par l’usage de nouvelles technologies elle en perfectionne la maîtrise ce qui conduit a une plus grande acceptation de la part des utilisateurs et donc une amélioration du produit. 
			L’apprentissage de l’entreprise est également dépendant de sa capacité à interagir avec les autres acteurs, en particulier ses fournisseurs et ses clients, afin de travailler ensemble dans l’adaptation aux nouvelles technologies.
			
			C’est cet apprentissage qui conduit l’entreprise à développer une base de connaissance qui servira tout au long du processus de recherche et d’innovation.
			
		\subsection{Les acteurs à prendre en compte}
			Les idées novatrices peuvent apparaître partout, et si les grosses entreprises ont les moyens d’investir dans de la recherche et du développement, ce n’est pas le cas des PME. C’est pourquoi les pôles de compétitivité, tel que Valorial, ont un rôle très important dans le processus d’innovation. Ils commencent tout d’abord par sélectionner les projets sur certains critères. Le Pôle Valorial spécifie dans ses critères d’éligibilité\cite{Eligibilite} qu’un projet pour être valide doit être dans une des thématiques du pôle, à savoir : “Technologies innovantes / Sécurité alimentaire / Nutrition santé animale ou humaine / Ingrédients fonctionnels”, qu’il doit être collaboratif, c’est à dire bénéficier de partenaires industriels et académiques, et qu’il doit être “source de création de valeur et de croissance pour les partenaires”.
			
			Une fois un projet sélectionné, le pôle de compétitivité accompagne l’entreprise dans le processus de recherche et de développement, et la conseil dans les démarches à réaliser. En particulier le pôle de compétitivité permet de trouver des financements via différents fonds : majoritairement l’État mais également la région / le département, les fonds d’investissement, les banques…
			A travers le pôle de compétitivité, l’entreprise peut trouver les compétences nécessaires à la réalisation de son projet.
				
		\subsection{La mise en place du produit}
			
	\section{Les bénéfices des l'innovation}
		
		\subsection{Une image revalorisée}
				
		\subsection{De nouvelles opportunités}
				
		\subsection{La création d'un nouveau marché}

\chapter{... mais qui ne réussit pas toujours}

	\section{Une innovation pas vraiment optimale}
	
		\subsection{Une vision trop portée sur le produit}
			Au delà des raisons historiques qui avaient pour but de relancer l’autosuffisance du pays, l’agriculture et toute l’industrie agro-alimentaire Bretonne ont subi depuis quelques décennies une modernisation massive. En effet, la région Bretonne ne disposant que de peu de terres arables, le développement de l’agriculture hors-sol a été l’un des développement phare de la région. 
			
			Dans la foulée, une majeure partie des techniques de génétique améliorée ont été développées en Bretagne. Au final, la greffe a pris puisque cette région centralise 30 \% des volailles et 50 \% des porcs consommés au sein de l’hexagone, ce qui représente une part important de l’économie de la région et est presque devenu symbolique.
			
			Mais ce développement s’est fait au mépris de l’environnement, les quotas de pêche et les considérations environnementales ont donné un sérieux coup de frein, notamment le taux de nitrates et l’arrivée des algues vertes qui ont stoppé l’extension des élevages.
			
			Ainsi, devant cette problématique d’expansion, les élevages se sont spécialisé dans le modèle dit “en batterie”,(IMG POULE BATTERIE) qui consiste à optimiser le nombre de bêtes élevées dans un environnement donné. Ces élevages sont vivement critiqués par les associations de défense des animaux  et ont été soumis à des législations dans certains pays comme la Suisse ou l’Allemagne. En Bretagne, seules les directives européennes font loi et les élevages n’en tiennent pas toujours compte, comme en témoigne cet article tiré de 30 millions d’amis\cite{BretagnePoussinsBroyesEtouffesDansCouvoir} : “Des poussins étouffés dans des sacs-poubelle, broyés vivants ou agonisants dans les bennes à ordures. C'est ce que révèle une vidéo publiée ce mercredi 12 novembre 2014 par l’organisme de protection animale L214.”
			
			Dans le cadre de notre travail avec le pôle de compétitivité Valorial, nous avons pu constater d’après leur base de données (BDD) recensant tous leurs projets traités jusqu’à maintenant, que la majeure partie de ces innovations concernaient la sécurité des consommateurs. 
			
			En effet, selon la BDD, 27\% des projets ont pour thème la santé et la nutrition. Ajoutant à ce chiffre les 32\% des projets  organisés autour de la qualité et de la sécurité des aliments, nous obtenons au total 59\% de projets qui ont pour but le bien-être du consommateur, laissant alors les autres thèmes se partager les 41\% restant. Nous devons aussi annoncer que sur les 10 projets les plus financés, 9 font partie des deux thèmes précédents, montrant ainsi l’importance accordée à la sécurité des consommateurs par les différents financeurs publics ou privés des projets de Valorial.
						
		\subsection{Un processus d'innovation parfois sans prise de recul}
			
			Nous avons expliqué précédemment la plus grande tendance de l’innovation dans l’industrie agro-alimentaire Bretonne, à savoir les innovations portées sur le produit, sa production, sa qualité, sa transparence. Peut-être que ce n’est pas ce qu’il faut à la région pour sortir de la crise.
			
			En effet, cette ligne de développement productiviste a causé des ravages au niveau environnemental. Les pesticides et les engrais chimiques utilisés massivement dans les cultures de la région déversent du nitrate dans les sols et ainsi les eaux. En trente ans, le taux de nitrates dans les rivières Bretonne a doublé, rendant parfois l’eau impropre à la consommation et participant à la prolifération d’algues vertes\cite{NitratesAlguesVertesBretagne}. Ces algues représentent un risque pour la santé humaine, leur décomposition entraine le rejet d’hydrogère sulfuré, qui est un gaz toxique. Il faut s’avoir qu’il s’en échouent entre 40 et 70 000 m3 chaque année et que le coût du ramassage s’élève à près de 500 000\euro.
			
			Les agriculteurs sont blamés car ils utilisent des engrais chimiques sur lesquels ils subissent des restrictions mais c’est le modèle économique qui veut celà. En effet, ce sont les agriculteurs qui en sont les premières victimes avec un salaire annuel moyen de 12 500\euro, l’avant denière position en terme de rémunération dans toute la France\cite{AlguesVertesNouvellePreuveRavagesProductivismeAgricole}. Malgré l’intensité de production, parfois au détriment des lois sanitaires et de protection des animaux, les agriculteurs peinent à gagner correctement leur vie.
			
			
			Le remembrement, qui est une innovation un peu plus ancienne (années 60-70) et qui consistait à rassembler et regrouper des terres cultivables pour les agriculteurs afin de les rendre plus accessibles et plus vastes, a aussi causé des ravages environnementaux. En effet, ce sont des milliers de kilomètres de haies et de talus qui ont été rasés et qui ont perturbé l’écosystème. Cette pratique a causé :
			\begin{itemize}
				\item des innondations
				\item parfois des torrents de boue
				\item l’assèchement de certains point d’eau
				\item la destruction du paysage de la région : le bocage, qui est la séparation des champs par des haies ou des talus de terre
				\item l’érosion des sols et leur appauvrissement
			\end{itemize}
			
			Autant de conséquences désastreuses sur l’environnement.	
			
			Mais au delà des innovations destructrices de l’environnement, les innovations développés dans la partie précédente, à savoir celles qui consistent à agir sur le produit en lui même, sa productivité, ses apports, la sécurité alimentaire, ne sont peut-être pas la bonne solution au problème économique du secteur.
			
			La consommation de viande de la région est plus coûteuse que celle importée, malgré les coûts de transport. Cette constatation amène une question qui peut se résumer ainsi : ne vaudrait-il pas mieux produire une viande de qualité et la vendre légèrement plus chère plutôt que de continuer à lutter contre les importation de viande de basse qualité et à faible coût alors qu’il est presque impossible de rivaliser ?
			
			La communication laisse également à désirer dans le secteur de l’agro-alimentaire de façon générale. Nous voyons très peu de publicité sur les produits de la région, quelques entreprises sont présentes comme les poulets loué qui essayent justement de répondre à la question précédente et vantent leur viande de qualité, qui se vend plus cher mais à juste prix.
			Certains produits étant délaissés, comme par exemple l’artichaut, les entreprises se lancent dans des campagnes de publicité décalés pour attirer l’oeil du consommateur, par exemple la publicité très “hot” de Prince de Bretagne.
			
		\subsection{L'agro-alimentaire, une industrie retardataire}
			Lorsque l’on pense au monde de l’agro-alimentaire, la première pensée que l’on a se rapporte majoritairement à l’agriculture. Qui dit agriculture, dit ferme, fermier, boue, charette et moissonneuse batteuse, dans l’esprit populaire.
			
			Malgré la modernisation colossale de l’agriculture au cours des 40 dernières années, ce stéréotype du monde agricole n’a pas beaucoup changé, pourtant on peut comparer certaines exploitations agricoles à de vraies usines, aux normes d’hygiène très strictes. Pourquoi cette image n’a-t-elle pas changé ?
			
			Tout d’abord, il faut resituer le contexte historique, plus particulièrement le contexte d’après guerre dans les années 50-60. La Bretagne était alors une région marginalisée et très en retard au niveau économique. L’électricité n’y est pas complètement installée et les voies de transport, particulièrement les voies ferrés, sont vétustes. L’agriculture est archaïque, ce sont de petites exploitations de subsistances très peu mécanisés.
			
			Ce n’est qu’au retour de la guerre que les paysans Bretons, enrolés dans des fermes allemandes, modernes et productivistes, vont se rendre compte du retard de leurs exploitations. Cette prise de conscience alliée au besoin de relancer l’économie et la production agricole du pays a permis de débloquer des aides considérables pour la région. Ainsi, le socialiste breton François Tanguy-Prigent, ministre de l'Agriculture dans le gouvernement d'union nationale du général de Gaulle, crée un statut pour le fermage et encourage le développement des organisations paysannes. L'Église n'est pas en reste, via la puissante Jeunesse agricole chrétienne (JAC), très implantée dans la péninsule. La JAC va entreprendre un effort sans précédent d'éducation et de formation des paysans. Elle fournit aussi des cadres qui se révéleront d'habiles négociateurs. Il faut enfin souligner le rôle du Comité d'études et de liaison des intérêts bretons (Célib) qui travaille à désenclaver la Bretagne et développer son territoire\cite{ModernisationAgricultureBretonne}. Entre 1962 et 1975, la population agricole de la région a triplé, alors qu’elle a à peine doublé en France.
			
			Cette modernisation de l’agriculture a été très bénéfique pour le secteur de l’agro-alimentaire Breton, plaçant la région en première position au niveau de la production grâce à ses infrastructures très productivistes comme l’élevage en batterie. Mais malgré tous ces efforts, les petites infrastructures peinent à se mettre à la page, surtout au niveau des innovations récentes. 
			
			En effet, bon nombre d’agriculteurs sont regroupés au sein de coopératives mais ils sont encore nombreux à ne pas utiliser l’outil informatique, devenu une des bases du système économique quel que soit le secteur d’activité. Il n’y a pas que le fait d’informatiser ses revenus, sa production ou consulter les prévisions météorologiques, cet outil permet de consulter les tendances des réseaux sociaux, de créer un site web pour son entreprise voire de proposer des ventes en ligne comme le site \url{http://www.bienvenue-a-la-ferme.com/} qui permet à tous de trouver le producteur le plus proche de chez soi.
			
			C’est autour de cette problématique que se sont développés certaines entreprises comme \url{http://www.mychefcom.com/} qui sont spécialisés dans l’accompagnement des entreprises agro-alimentaire dans leur campagne de développement digital. Monsieur Le Bayon, Community Manager dans l’entreprise Mychefcom citée précédemment, nous a par exemple raconté que certaines grandes entreprises avec qui ils travaillent lui ont bien précisé de contacter le producteur par téléphone ou par courrier car il ne disposait pas d’un ordinateur personnel ou professionnel. raconté que certaines grandes entreprises avec qui ils travaillent lui ont bien précisé de contacter le producteur par téléphone ou par courrier car il ne disposait pas d’un ordinateur personnel ou professionnel. Ces entreprises proposent donc des services comme le consulting en stratégie digitale, qui est l’accompagnement de l’entreprise dans la mise en place de sa stratégie commerciale sur le web ,la création d’un site web, une liste de contacts mail ou bien d’implanter l’entreprise sur les réseaux sociaux.
			
			Nous pouvons donc constater que le domaine de l’agro-alimentaire est, malgré certaines structures innovantes comme le pôle de compétitivité Valorial, un secteur en retard au niveau de l’innovation. Pour autant, cette tendance commence à s’inverser et de plus en plus d’entreprises et de producteurs se lancent dans l’exportation à l’étranger, la modernisation de leur communication et s’ouvrent petit à petit au monde de l’industrie moderne.
			
	\section{L'innovation, inaccessible pour certains}
		
		\subsection{Les contreparties à l'innovation}
			
		\subsection{Le coût élevé}
				
		\subsection{Une vision de spécialisation pour les petites structures}
			
	\section{Les solutions actuelles}
		Aujourd’hui, plusieurs solutions existent pour améliorer la pratique de l’innovation dans les entreprises. Afin de mieux cibler ce qui peut être fait, nous commencerons par étudier l’avantage qu’apporte les pôles de compétitivité et ce qu’il peuvent amener à l’avenir. Ensuite, nous verrons les nouveaux processus d’innovations adaptés à l’agro-alimentaire. Pour terminer, nous nous pencherons sur les avancées effectuées en matière de recherche scientifique.
		
		\subsection{Les pôles de compétitivité}
		Tout d’abord, les pôles de compétitivité constituent un facteur important en ce qui concerne l’innovation.
		
		Pour commencer, il est important de rappeler la spécificité des pôles de compétitivité, organisations créées il y a dix ans suite au Rapport Blanc donnant en 2004 la priorité à la croissance et à la compétitivité dans l’économie nationale\cite{RapportBlanc}. Il existe quatre types de systèmes collaboratifs, se différenciant selon qui souhaite la collaboration et qui collabore, comme illustré ci-dessous\cite{PoleCompetitivite}.
		
		\begin{tabularx}{\linewidth}{|X|X|X|}
			\hline
			& Collaboration voulue par les acteurs eux-mêmes & Collaboration reconnue et renforcée par les pouvoirs publics\\
			\hline
			Tous les partenaires sont des entreprises & Districs industriels & Systèmes Productifs Locaux (SPL)\\
			\hline
			Partenaires d'organisations variées (entreprises, universités, ...) & Cluster & Pôles de compétitivité\\
			\hline
		\end{tabularx}
		\emph{Les pôles de compétitivité, une forme de collaboration inter-organisationnelle\\}

		Ainsi, les pôles de compétitivités diffèrent de leurs semblables par le large éventail d’organisations concernées, mais aussi par son origine gouvernementale.
		Ensuite, il est intéressante de savoir comment fonctionne un pôle de compétitivité. Suite à une étude menée par Thomas Froehlicher et Franck Barès sur plusieurs agglomérations et régions, dont la Bretagne, a permis de dessiner un fonctionnementcite{PolesCompetitiviteClusters}. Dans un premier temps, les pôles de compétitivité se renseignent au travers de veilles. Grâce à cela, ils deviennent capables d’anticiper les futures tendances. Ils peuvent donc projeter leur activité dans ce futur et se restructurer en fonction de cela. Le pôle de compétitivité qui a suivi ces étapes devient alors capable d’inventer les produits de demain. Ces produits sont ensuite fabriqués par les entreprises à l’origine de projets.
		Ces pôles sont d’ailleurs, pour la plupart, relativement efficaces. Sur les 71 pôles de compétitivité français, 39 ont réussi à respecter leurs objectifs, tandis que seulement 13 devaient être restructurés en 2008\cite{PolesCompetitiviteClusters}. Pour le pôle de compétitivité breton Valorial, ce sont pas moins de 250 projets qui ont été labellisés en moins de dix ans\cite{ProjetsValorial}. Parmi eux, on trouve le projet SpectraG qui a permis à l’entreprise Valorex de mettre en place une méthode pour prédire les acides gras provenant des tissus adipeux dans les élevages, et donc d’améliorer la qualité des produits en termes de nutrition\cite{ProjetsAboutis}. Un nombre conséquent d’autres projets ont pu être menés à rien en partie par l’action de pôles de compétitivité sur toute la France.
		Pour autant, ce ne sont pas les pôles de compétitivité qui ont permis aux entreprises d’entrer en contact les unes avec les autres. En effet, un tissu industriel était déjà présent avant 2004. Malgré tout, la création de ces pôles aura permis le renforcement de ce phénomène. On voit par ailleurs que, depuis leur création, les pôles de compétitivité sont passés d’une optique plutôt gouvernementale à une place de catalyseur dans l’économie locale.
		D’autant plus, les actions de ces pôles sont parfois limitées, et le manque d’influence sur la distribution des ressources se fait sentir. On constate par exemple que certaines grosses entreprises sont capables de se passer de l’aide des pôles sans avoir pour autant une réelle organisation dans l’innovation, tandis que des petites structures innovantes peuvent se retrouver bloquées à cause du manque de ressources qui leurs sont allouées par les partenaires.
		C’est pour cela que les pôles de compétitivité tendent à se rapprocher d’un écosystème de croissance, qui permettrait d’avoir une plus grande emprise sur les ressources qui passent par l’organisation, et de renforcer encore plus efficacement l’innovation.
		
		En conclusion, les pôles de compétitivité revêtent une importance toute particulière au sein de l’économie, en permettant aux entreprises d’innover plus et mieux.
				
		\subsection{Les nouveaux modèles d’innovation}
			Suite aux problèmes qu’à l’industrie agro-alimentaire à se renouveler, plusieurs modèles ont été mis aux points dans le but de faciliter l’innovation dans un secteur considéré comme mature et à faible taux d’innovation.
			
			En premier lieu, il convient d’expliquer d’où provient le besoin en termes de nouveaux modèles d’innovation. Dans l’économie actuelle, l’industrie agro-alimentaire est considérée comme une industrie mature, puisqu’elle n’offre plus beaucoup de perspectives d’évolution. En réalité, la plupart des innovations qui peuvent servir cette industrie ont déjà été produites. C’est pour cela que l’innovation paraît depuis quelques années être peu avantageuse pour les entreprises du secteur. Cela a produit un désintéressement de la part de beaucoup d’acteurs, faisant ainsi de cette industrie une industrie à faible taux d’innovation. Or, il est aujourd’hui possible d’apporter des innovations importantes et rentables, comme l’ont montrés de nombreuses entreprises.
			Dans l’optique de relancer l’innovation dans l’industrie agro-alimentaire, trois principaux modèles ont été proposés, tous dans une vision d’Open Innovation (innovation plus ouverte et collaborative)\cite{OpenInnovation}. Le premier d’entre eux est Sharing is Winning (littéralement, “Partager c’est Gagner”)\cite{SiW}. Ce modèle se base sur la coopération entre partenaires de natures diverses pour réussir. Même s’il est utilisé depuis quelques années, ce modèle est défini par Traitler et Saguy en 2009. Il a pour but de partir d’innovations proposées par les consommateurs, puis de les dispatcher sur toute la filière de production (fournisseurs et acheteurs compris)\cite{OIFr}. Un exemple d’entreprise ayant adopté ce modèle est le groupe Nestlé\cite{NestleOI}, qui depuis 2007 met en place ce mode de fonctionnement en partenariat avec Cargill, BASF et l'Université Polytechnique de Lausanne. Le processus d’innovation s’en est trouvé accéléré de manière sensible.	
			
			Un autre modèle proposé est le food-machinery framework (littéralement, “Structure du mécanisme alimentaire”). Institué par Bigliardi, Bottani et Galati, il s’intéresse à la chaîne de production et aux acteurs qui l’entourent\cite{FMF}. Il est assez facile à mettre en œuvre pour de grosses structures, puisqu’elles ont par avance un réseau leur permettant de collaborer plus simplement et plus souvent.
			
			Enfin, le modèle Want, Find, Get, Manage (“Vouloir, Chercher, Obtenir, Gérer”) se veut très ouvert sur le fonctionnement existant. Ce modèle, initialement proposé par Slowinski\cite{WFGM}, propose de s’intéresser, quel que soit le problème, à ce que d’autres acteurs ont déjà mis en place. Contrairement au précédent, ce modèle est plus accessible pour les petites structures, puisqu’il suffit de se renseigner sur ce qui existe plutôt que de butter sur un problème. Il est même préférable de ne pas avoir trop de relations à gérer dans ce modèle.
			
			Ainsi, même si l’industrie agro-alimentaire peine à se renouveler, il existe aujourd’hui des solutions, gravitant autour de l’Open Innovation. Il apparaît donc que l’industrie agro-alimentaire nécessite avant tout la mise en place d’un collaboration.
			
		\subsection{Les avancées en matière de recherche}
			Les solutions précitées ne sont bien évidemment pas les seules à pouvoir redresser la barre de l’innovation dans l’industrie agro-alimentaire. En parallèle avec celles-ci, plusieurs autres pistes existent. Nous commencerons par voir ce qui est apporté par les autorités publiques. Nous parlerons ensuite des nouvelles avancées dans la recherche.
			
			Pour commencer, plusieurs solutions pour améliorer l’innovation dans le secteur ont été proposées par les pouvoirs publiques. L’OCDE a présenté, en 2009, un rapport \cite{OECD} faisant suite à des observations faites sur la période 2007-2008. Même si certaines de ces ouvertures ont déjà été exploitées (les bioproduits, par exemple, représentent aujourd’hui un marché en vogue). Pour autant, des domaines comme celui des aliments fonctionnels (aussi appelés alicaments) restent encore trop peu utilisés. Ainsi, même si quelques entreprises portent des projets de cet intérêt, elles sont une exception. Cela s’explique par le fait qu’il est difficile de créer des aliments avec un effet bénéfique sur la santé, et que le coût de ces aliments est élevé. Pourtant, la filière des aliments fonctionnels représente à la fois un bénéfice pour la santé publique, elle est également très fructueuse pour les entreprises (75,5 milliards de dollars de la création à la fin du siècle dernier jusqu’à 2009). Cette filière, lancée en France par Unilever et sa margarine réduisant le taux de cholestérol (encore vendue sous le nom de Fruit d’Or ProActiv)\cite{FruitDOr}, est donc l’une des priorités publiques depuis quelques années en matière d’innovation agro-alimentaire. Le gouvernement français se penche également de manière significative sur le problème de la communication dans l’industrie agro-alimentaire. Ainsi, il propose de valoriser les produits français avec les logos Viande de France cités plus tôt, mais cherche aussi à instituer une simplification à l’égard des consommateurs avec la notation par étoiles sur les pièces bouchères\cite{EtoilesViande}. 
%PHOTO ALICAMENT

			Ensuite, puisque l’innovation est un problème majeur dans l’actualité, une importance toute particulière lui est donnée par l’État, ce qui a mené entre autres à la création de pôles de compétitivité axés sur l’agro-alimentaire, au nombre de 10 sur les 71 pôles labellisés dans le pays\cite{CompetitiviteGouv}. Cet appui donné aux pôles de compétitivité agro-alimentaires crée donc une synergie, comme expliqué plus tôt. Mais cette synergie s’ajoute aujourd’hui à la prise de conscience de nombreux acteurs, ce qui améliore de manière considérable l’innovation. Il reste néanmoins à sensibiliser tous les acteurs de manière efficace et définitive, dans l’optique d’obtenir un fonctionnement réellement adapté pour la mise en place d’une stratégie d’innovation au niveau national.

			Enfin, la recherche propose elle aussi des solutions d’avenir. En se tournant vers de nouveaux domaines, plus porteurs, tels que la qualité des aliments, la santé et la nutrition. Ainsi, le pôle agronomique cherche à trouver des méthodes plus respectueuses de l’environnement, et permettant la production la plus écologique possible\cite{PoleAgroOuest}. Tout cela a pour intérêt de renforcer à la fois la qualité du produit (en terme de risques sanitaires et de nutrition). Les centres de recherche comme le pôle agronomique et l’INRA (Institut National de la Recherche Agronomique) se mettent également à développer des réseaux dans le but d’accompagner les entreprises et l’industrie qu’elles portent de façon plus efficace. À cette fin, le pôle agronomique de l’Ouest est en relation avec les pôles de compétitivité Valorial en Bretagne et Végépolys en Pays de Loire, tout en participant activement à la mise du pôle Mer\cite{PoleAgroOuest}. 

  			 Pour conclure, on remarque que tous les acteurs ont un rôle à jouer dans la mise en place d’une réelle innovation au sein de l’industrie agro-alimentaire.

\chapter*{Conclusion}
\addcontentsline{toc}{chapter}{Conclusion}

\bibliography{Mono}{}
\bibliographystyle{unsrt} %unsrt pour ranger par ordre d'apparition

\end{document}
