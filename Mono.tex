\documentclass[a4paper,10pt]{report}
\usepackage[utf8x]{inputenc}
\usepackage[T1]{fontenc}
\usepackage[french]{babel}

\title{La place de Valorial dans l'économie bretonne}
\author{ \bsc{Aurélien Fontaine}
	\and \bsc{Manuteau Huang} 
	\and \bsc{Nicolas Le Borgne}
	\and \bsc{Maxime Cadoret}
	\and \bsc{Flavien Lecuyer}
}

\begin{document}
	\maketitle
	\setcounter{tocdepth}{2}
	\tableofcontents
	
\chapter*{Introduction}
\addcontentsline{toc}{chapter}{Introduction}
	 L’industrie agro-alimentaire tient depuis longtemps une place prépondérante dans l’économie bretonne. En effet, celle-ci représente plus de 40 du secteur industriel régional ainsi que 15\% de la valeur ajoutée créée pour ce secteur au niveau national. 
	 
	 L’importance de l’agro-alimentaire dans le tissu économique breton rend d’ailleurs de la région une image très rurale, laquelle tend à être préservée tant l’évolution de ce secteur est significative. Malgré la baisse conséquente du nombre de salariés dans le secteur manufacturier entre 2010 et 2012, qui ont ainsi été réduit de 42\%, l’agro-alimentaire reste la principale source d’emploi dans la région. La part de l’agro-alimentaire dans la masse salariale de l’industrie manufacturière est passée à près de deux emplois sur trois (37 à 62\%).
	 
	 Malgré tout, l’économie agro-alimentaire ne garde qu’une importance faible dans l’apport de richesses au territoire local. La prédominance des industries de première transformation provoque ainsi une baisse des taux de valeur ajoutée par rapport à la moyenne nationale des IAA (Industries Agro-Alimentaires). Cela fragilise ainsi cette économie, qui doit donc réduire les prix afin de rester compétitive face aux marchés étrangers tout en luttant contre les lois françaises de plus en plus strictes. A titre d’exemple, les années 2012 et 2013 ont été marquées par des restructurations importantes touchant essentiellement le secteur de l’abattage et de la première transformation. Ces restructurations, qui pourraient se poursuivre, ont eu des conséquences sur l’emploi dans des territoires fragilisés, notamment dans le Morbihan et dans le Finistère. Les difficultés subies dans l’agro-alimentaire, en particulier pour les filières porcines et volaillère, ont créé des réactions en chaîne, qui ont de ce fait dépassé le cadre de l’agroalimentaire.
	 
	 Dans le but de mieux répondre à ces problèmes, on constate de la part des entreprises une volonté forte de se diversifier au travers de l’innovation. En l’espace de dix ans, les industriels ont ainsi réussi à mettre en place de nouvelles gammes de produit, permettant donc d’augmenter leur taux de valeur ajoutée et par extension placer la Bretagne à la première place en termes de valeur ajoutée dans le classement français. C’est dans cette optique que le pôle de compétitivité Valorial intervient. Le but de cette organisation est de mettre en relation industriels, financeurs et laboratoires de recherches afin d’apporter leur aide aux différents acteurs qui cherchent à donner le jour à des projets innovants. Chaque projet est étudié et labellisé ou non suivant des critères spécifiques et se voit accorder des subventions de l’état ou de financeurs privés, voire des deux, à hauteur de leurs besoins et de la décision de Valorial.
	 
	 Cela donne par conséquent une place spécifique à Valorial dans l’économie bretonne bien qu’elle ne soit qu’une petite structure avec une dizaine d’employés. Pour autant, au fil des ans, les subventions de l’état s’amenuisent et mettent en péril l’avenir de l’organisation. Nous pouvons ainsi nous demander « Quelle est la place du pôle de compétitivité Valorial dans l’économie bretonne et comment peut-il évoluer à l’avenir ? » Afin de répondre à cette question, nous nous intéresserons de prime abord à la situation de l’agro-alimentaire dans la région Bretagne. Dans un second temps, nous présenterons le pôle de compétitivité et ses actions dans le grand ouest. Pour finir, nous nous pencherons sur les perspectives d’avenir de Valorial.
	 
\chapter{La situation agro-alimentaire bretonne}
	Aujourd’hui, et depuis quelques années, la situation de l’industrie agro-alimentaire tend à se dégrader à l’international. Pour autant, la Bretagne semble bien s’en sortir, comme en attestent les chiffres. Cela rend donc la situation de cette industrie particulière. Afin de la présenter en détail, nous commencerons par voir la situation de l’agro-alimentaire français à l’international. Ensuite, nous comparerons la situation de la Bretagne à celle de la France. Pour finir, nous parlerons de la crise dans l’agro-alimentaire et des actions menées pour la gérer.

	\section{L’agro-alimentaire français dans le marché international}
		Comme beaucoup le savent, l’agro-alimentaire est dans une situation peu avantageuse depuis quelques années, affaiblie par plusieurs crises en quelques années. Afin de donner un état des lieus de cette industrie, nous verrons tout d’abord les problèmes qu’ont les entreprises à conserver leur compétitivité. Dans un second temps, nous nous intéresserons aux problèmes de l’emploi dans le secteur. Pour finir, nous verrons que ce mauvais bilan se rattrape au travers de la technologie et de l’export.

		\subsection{La compétitivité en baisse}
			Pour commencer, on remarque que la compétitivité de l’industrie agro-alimentaire est dans une tendance à la baisse.

		\subsection{Le paradoxe de la productivité et de l'emploi}
			Pour ce qui est de l’emploi, on constate aujourd’hui un paradoxe important. En effet, les industriels de l’industrie agro-alimentaire sont de plus en plus productifs, mais on se heurte en parallèle avec une baisse de plus en plus conséquente du nombre d’emplois et d’organisations dans le secteur.

		\subsection{L'export et la technologie de pointe, des activités qui portent}
			Malgré ce bilan plutôt négatif en ce qui concerne l’agro-alimentaire français, il reste à souligner les points positifs dans l’évolution de ces dernières années. En effet, les entreprises qui se tournent vers l’export et la technologie de pointe arrivent à garder une place importante sur le marché, voire même à se développer.

	\section{La Bretagne : un cas particulier}
	Nous avons donc vu que la situation de l’industrie agro-alimentaire en France est de moins en moins bonne. Pour autant, elle semble avoir un bon ancrage dans l’économie bretonne. De manière à expliquer les caractéristiques de ce secteur en Bretagne, nous étudierons pour débuter la place de la Bretagne en France, aussi bien géographiquement qu’économiquement. Deuxièmement, nous parlerons de la place importante de l’industrie agro-alimentaire en Bretagne. Pour terminer, nous étudierons les secteurs d’activités de l’agro-alimentaire en Bretagne.

		\subsection{La place de la Bretagne en France}
			L’une des raisons pour lesquelles l’industrie agro-alimentaire se porte mieux en Bretagne qu’en France s’explique par la position de la région, aussi bien économiquement que géographiquement. En effet, la région présente de multiples spécificités qui font d’elle un cas à part.
	
			En effet, la Bretagne possède depuis longtemps un tissu rural très présent, de la tradition des goémoniers aux éleveurs de porcs la Bretagne a, pour le reste de la France, une image campagnarde indépendante. Mais en plus de son histoire avec l’agriculture, la Bretagne s’est développé dans ce sens grâce à des spécificités géographiques. Nous pouvons voir grâce à ces cartes (figure 1) la répartition de la population dans la région. Le massif armoricain étant une chaine de montagne de 65 000 km² culminant à un peu plus de 400m d’altitude, il a pour principal relief de vastes landes escarpées où la seule activité subsistante est l’agriculture et ses vastes pâturages.
	
			A cause de ces reliefs, nous pouvons aussi constater que la population de la Bretagne est localisée principalement proche des littoraux ou bien du bassin Rennais (figure 2). Cette répartition a conduit au développement des littoraux Bretons au niveau portuaire. Les trois principaux ports de trafic de marchandises étant Saint-Malo, Brest et Lorient qui ont réalisés en 2012 84\% du trafic total de la région, pour un total de 8.297 millions de tonnes aux trois quarts destinés à l’Union Européenne ou au reste de la France. Ce sont également plus de 3.9 millions de passagers qui ont  transité par les ports de la péninsule Bretonne au cours de cette année 2012. On peut donc constater un développement important du commerce maritime Breton qui, en majeure partie, est occupé par les produits agricoles et alimentaires (35 \%), devant les produits énergétiques (24 \%) puis la métallurgie (18 \%).
	
			Mais même si la Bretagne se tourne vers l’extérieur grâce à ses échanges maritimes et que les deux tiers du trafic total de marchandise sont des échanges intra régionaux à hauteur de 120 millions de tonnes, les échanges nationaux représentent 49 millions de tonnes de marchandises donc 42\% sont des produits agricoles et alimentaires. Ces échanges se font principalement avec les régions limitrophes de la Bretagne (90 \%). En effet, en tant que première région agricole Française en termes de production, le niveau d’importation de denrées alimentaire reste limité à celles que le climat régional rend impossible à cultiver. Le fait que la Bretagne soit également la première région Française au niveau de la pêche et que l’élevage soit une part importante de la production agricole, rend l’importation de ces produits une pratique très rare.
	
			Au travers de cette présentation de la Bretagne et de la place qu’occupe cette région en France, nous avons pu constater que la tendance historique d’indépendantisme Breton était liée à sa capacité à produire la majeure partie de ses produits de consommation en tant que première région agricole de France. Les échanges hors de la région sont principalement des exportations, par exemple en 2006, les exportations de la Bretagne représentaient 8.9 milliards d’euros contre 7.8 milliards d’euros en importation. La Bretagne occupe donc une place importante dans la production agricole Française, qui était en 2012 considérée comme la 2e puissance agricole mondiale.
	
		\subsection{L'importance de l'agro-alimentaire dans l'industrie bretonne}
			L’industrie agro-alimentaire a une place réellement privilégiée dans l’industrie bretonne. Ainsi, elle représente quatre salariés sur dix dans l’industrie bretonne.
			
		\subsection{De multiples secteurs d'activités}
			Enfin, on peut expliquer la situation de l’agro-alimentaire en Bretagne par la multiplicité des secteurs d’activités existants dans la région. En effet, non seulement cette industrie est importante dans l’économie bretonne, mais elle se présente aussi sous plusieurs formes.
			
	\section{La crise, impact et réponses}
		Bien que la Bretagne s’en sorte bien dans la crise de l’agro-alimentaire, cette dernière reste importante et pose de nombreux problèmes. Dans l’optique de savoir ce qui se passe autour de cette crise, nous commencerons par comparer ses impacts au niveau national et au niveau breton. Nous verrons ensuite quelles sont les actions menées par les pouvoirs publics. Enfin, nous parlerons des solutions envisagées pour l’avenir.
		
		\subsection{Des conséquences différentes entre la France et la Bretagne}
			Avant de pouvoir parler de la réaction générale face à la crise, regardons pour commencer ses impacts. Afin de mieux se rendre compte de la différence des conséquences en France et en Bretagne, nous procéderons tout d’abord à une analyse de l’impact aux différentes échelles, puis nous comparerons les résultats.
			
		\subsection{Différentes actions à différentes échelles}
			En réponse à cette crise, les pouvoirs publics ont mis en œuvre de nombreuses actions, plus ou moins efficaces. Mais ces actions ne sont pas toutes venues du même échelon. En effet, certaines d’entre elles ont été menées au niveau national, d’autre à des niveaux plus locaux. Afin de mieux s’en rendre compte, nous commencerons par voir les actions du gouvernement. Ensuite nous verrons ce qui a été fait par la région et les départements.
			
		\subsection{Quelles solutions pour l'avenir ?}
			Enfin, il est important de se rendre compte des solutions envisagées pour l’avenir de l’agro-alimentaire. 
			
\chapter{L'innovation, un remède contre la crise ...}
	
\section{Les différentes formes d'innovation}
	
		\subsection{L'innovation sur les modes de production}
		
\subsection{L'innovation sur l'organisation des entreprises}
		
\subsection{L'innovation par les nouvelles ressources}
		
\section{Comment fait-on pour innover ?}
		
\subsection{La recherche au sein des entreprises}
		
\subsection{Les acteurs à prendre en compte}
		
\subsection{La mise en place du produit}
		
\section{Les bénéfices des l'innovation}
	
\subsection{Une image revalorisée}
		
\subsection{De nouvelles opportunités}
		
\subsection{La création d'un nouveau marché}

\chapter{... mais qui ne réussit pas toujours}

\section{Une innovation pas vraiment optimale}

\subsection{Une vision trop portée sur le produit}
		
\subsection{Un processus d'innovation parfois trop rapide}
		
\subsection{L'agro-alimentaire, une industrie retardataire}
	
\section{L'innovation, inaccessible pour certains}
	
\subsection{Les contreparties à l'innovation}
		
\subsection{Le coût élevé}
		
\subsection{Une vision de spécialisation pour les petites structures}
		
\section{Les solutions actuelles}
	
\subsection{Les pôles de compétitivité}
		
\subsection{Les nouveaux modèles d’innovation}
		
\subsection{Les avancées en matière de recherche}

\chapter*{Conclusion}
\addcontentsline{toc}{chapter}{Conclusion}

\end{document}
