\documentclass[a4paper,10pt]{report}
\usepackage[utf8x]{inputenc}
\usepackage[T1]{fontenc}
\usepackage[french]{babel}
\usepackage{graphicx}
\usepackage{eso-pic}
\newcommand\BackgroundPic{%
	\put(0,250){%
		\parbox[b][\paperheight]{\paperwidth}{%
			\vfill
			\centering
			\includegraphics[width=\paperwidth,height=\paperheight,keepaspectratio]{logoINSA.jpg}
			\vfill
}}}


\title{La place de Valorial dans l'économie bretonne}
\author{ \bsc{Aurélien Fontaine}
	\and \bsc{Manuteau Huang} 
	\and \bsc{Nicolas Le Borgne}
	\and \bsc{Maxime Cadoret}
	\and \bsc{Flavien Lecuyer}
}
%\institute[INSA de Rennes]{Institut National des Sciences Appliquées de Rennes}
\date{\today}



\begin{document}
	\AddToShipoutPicture*{\BackgroundPic}
	\maketitle
	\setcounter{tocdepth}{2}
	\tableofcontents
	
\chapter*{Introduction}
\addcontentsline{toc}{chapter}{Introduction}
	 L’industrie agro-alimentaire tient depuis longtemps une place prépondérante dans l’économie bretonne. En effet, celle-ci représente plus de 40 du secteur industriel régional ainsi que 15\% de la valeur ajoutée créée pour ce secteur au niveau national. 
	 
	 L’importance de l’agro-alimentaire dans le tissu économique breton rend d’ailleurs de la région une image très rurale, laquelle tend à être préservée tant l’évolution de ce secteur est significative. Malgré la baisse conséquente du nombre de salariés dans le secteur manufacturier entre 2010 et 2012, qui ont ainsi été réduit de 42\%, l’agro-alimentaire reste la principale source d’emploi dans la région. La part de l’agro-alimentaire dans la masse salariale de l’industrie manufacturière est passée à près de deux emplois sur trois (37 à 62\%).
	 
	 Malgré tout, l’économie agro-alimentaire ne garde qu’une importance faible dans l’apport de richesses au territoire local. La prédominance des industries de première transformation provoque ainsi une baisse des taux de valeur ajoutée par rapport à la moyenne nationale des IAA (Industries Agro-Alimentaires). Cela fragilise ainsi cette économie, qui doit donc réduire les prix afin de rester compétitive face aux marchés étrangers tout en luttant contre les lois françaises de plus en plus strictes. A titre d’exemple, les années 2012 et 2013 ont été marquées par des restructurations importantes touchant essentiellement le secteur de l’abattage et de la première transformation. Ces restructurations, qui pourraient se poursuivre, ont eu des conséquences sur l’emploi dans des territoires fragilisés, notamment dans le Morbihan et dans le Finistère. Les difficultés subies dans l’agro-alimentaire, en particulier pour les filières porcines et volaillère, ont créé des réactions en chaîne, qui ont de ce fait dépassé le cadre de l’agroalimentaire.
	 
	 Dans le but de mieux répondre à ces problèmes, on constate de la part des entreprises une volonté forte de se diversifier au travers de l’innovation. En l’espace de dix ans, les industriels ont ainsi réussi à mettre en place de nouvelles gammes de produit, permettant donc d’augmenter leur taux de valeur ajoutée et par extension placer la Bretagne à la première place en termes de valeur ajoutée dans le classement français. C’est dans cette optique que le pôle de compétitivité Valorial intervient. Le but de cette organisation est de mettre en relation industriels, financeurs et laboratoires de recherches afin d’apporter leur aide aux différents acteurs qui cherchent à donner le jour à des projets innovants. Chaque projet est étudié et labellisé ou non suivant des critères spécifiques et se voit accorder des subventions de l’état ou de financeurs privés, voire des deux, à hauteur de leurs besoins et de la décision de Valorial.
	 
	 Cela donne par conséquent une place spécifique à Valorial dans l’économie bretonne bien qu’elle ne soit qu’une petite structure avec une dizaine d’employés. Pour autant, au fil des ans, les subventions de l’état s’amenuisent et mettent en péril l’avenir de l’organisation. Nous pouvons ainsi nous demander « Quelle est la place du pôle de compétitivité Valorial dans l’économie bretonne et comment peut-il évoluer à l’avenir ? » Afin de répondre à cette question, nous nous intéresserons de prime abord à la situation de l’agro-alimentaire dans la région Bretagne. Dans un second temps, nous présenterons le pôle de compétitivité et ses actions dans le grand ouest. Pour finir, nous nous pencherons sur les perspectives d’avenir de Valorial.
	 
\chapter{La situation agro-alimentaire bretonne}
	Aujourd’hui, et depuis quelques années, la situation de l’industrie agro-alimentaire tend à se dégrader à l’international. Pour autant, la Bretagne semble bien s’en sortir, comme en attestent les chiffres. Cela rend donc la situation de cette industrie particulière. Afin de la présenter en détail, nous commencerons par voir la situation de l’agro-alimentaire français à l’international. Ensuite, nous comparerons la situation de la Bretagne à celle de la France. Pour finir, nous parlerons de la crise dans l’agro-alimentaire et des actions menées pour la gérer.

	\section{L’agro-alimentaire français dans le marché international}
		Comme beaucoup le savent, l’agro-alimentaire est dans une situation peu avantageuse depuis quelques années, affaiblie par plusieurs crises en quelques années. Afin de donner un état des lieus de cette industrie, nous verrons tout d’abord les problèmes qu’ont les entreprises à conserver leur compétitivité. Dans un second temps, nous nous intéresserons aux problèmes de l’emploi dans le secteur. Pour finir, nous verrons que ce mauvais bilan se rattrape au travers de la technologie et de l’export.

		\subsection{La compétitivité en baisse}
			Pour commencer, on remarque que la compétitivité de l’industrie agro-alimentaire est dans une tendance à la baisse.

		\subsection{Le paradoxe de la productivité et de l'emploi}
			Pour ce qui est de l’emploi, on constate aujourd’hui un paradoxe important. En effet, les industriels de l’industrie agro-alimentaire sont de plus en plus productifs, mais on se heurte en parallèle avec une baisse de plus en plus conséquente du nombre d’emplois et d’organisations dans le secteur.

		\subsection{L'export et la technologie de pointe, des activités qui portent}
			Malgré ce bilan plutôt négatif en ce qui concerne l’agro-alimentaire français, il reste à souligner les points positifs dans l’évolution de ces dernières années. En effet, les entreprises qui se tournent vers l’export et la technologie de pointe arrivent à garder une place importante sur le marché, voire même à se développer.

	\section{La Bretagne : un cas particulier}
	Nous avons donc vu que la situation de l’industrie agro-alimentaire en France est de moins en moins bonne. Pour autant, elle semble avoir un bon ancrage dans l’économie bretonne. De manière à expliquer les caractéristiques de ce secteur en Bretagne, nous étudierons pour débuter la place de la Bretagne en France, aussi bien géographiquement qu’économiquement. Deuxièmement, nous parlerons de la place importante de l’industrie agro-alimentaire en Bretagne. Pour terminer, nous étudierons les secteurs d’activités de l’agro-alimentaire en Bretagne.

		\subsection{La place de la Bretagne en France}
			L’une des raisons pour lesquelles l’industrie agro-alimentaire se porte mieux en Bretagne qu’en France s’explique par la position de la région, aussi bien économiquement que géographiquement. En effet, la région présente de multiples spécificités qui font d’elle un cas à part.
	
			En effet, la Bretagne possède depuis longtemps un tissu rural très présent, de la tradition des goémoniers aux éleveurs de porcs la Bretagne a, pour le reste de la France, une image campagnarde indépendante. Mais en plus de son histoire avec l’agriculture, la Bretagne s’est développé dans ce sens grâce à des spécificités géographiques. Nous pouvons voir grâce à ces cartes (figure 1) la répartition de la population dans la région. Le massif armoricain étant une chaine de montagne de 65 000 km² culminant à un peu plus de 400m d’altitude, il a pour principal relief de vastes landes escarpées où la seule activité subsistante est l’agriculture et ses vastes pâturages.
	
			A cause de ces reliefs, nous pouvons aussi constater que la population de la Bretagne est localisée principalement proche des littoraux ou bien du bassin Rennais (figure 2). Cette répartition a conduit au développement des littoraux Bretons au niveau portuaire. Les trois principaux ports de trafic de marchandises étant Saint-Malo, Brest et Lorient qui ont réalisés en 2012 84\% du trafic total de la région, pour un total de 8.297 millions de tonnes aux trois quarts destinés à l’Union Européenne ou au reste de la France. Ce sont également plus de 3.9 millions de passagers qui ont  transité par les ports de la péninsule Bretonne au cours de cette année 2012. On peut donc constater un développement important du commerce maritime Breton qui, en majeure partie, est occupé par les produits agricoles et alimentaires (35 \%), devant les produits énergétiques (24 \%) puis la métallurgie (18 \%).
	
			Mais même si la Bretagne se tourne vers l’extérieur grâce à ses échanges maritimes et que les deux tiers du trafic total de marchandise sont des échanges intra régionaux à hauteur de 120 millions de tonnes, les échanges nationaux représentent 49 millions de tonnes de marchandises donc 42\% sont des produits agricoles et alimentaires. Ces échanges se font principalement avec les régions limitrophes de la Bretagne (90 \%). En effet, en tant que première région agricole Française en termes de production, le niveau d’importation de denrées alimentaire reste limité à celles que le climat régional rend impossible à cultiver. Le fait que la Bretagne soit également la première région Française au niveau de la pêche et que l’élevage soit une part importante de la production agricole, rend l’importation de ces produits une pratique très rare.
	
			Au travers de cette présentation de la Bretagne et de la place qu’occupe cette région en France, nous avons pu constater que la tendance historique d’indépendantisme Breton était liée à sa capacité à produire la majeure partie de ses produits de consommation en tant que première région agricole de France. Les échanges hors de la région sont principalement des exportations, par exemple en 2006, les exportations de la Bretagne représentaient 8.9 milliards d’euros contre 7.8 milliards d’euros en importation. La Bretagne occupe donc une place importante dans la production agricole Française, qui était en 2012 considérée comme la 2e puissance agricole mondiale.
	
		\subsection{L'importance de l'agro-alimentaire dans l'industrie bretonne}
			L’industrie agro-alimentaire a une place réellement privilégiée dans l’industrie bretonne. Ainsi, elle représente quatre salariés sur dix dans l’industrie bretonne.
			
		\subsection{De multiples secteurs d'activités}
			Enfin, on peut expliquer la situation de l’agro-alimentaire en Bretagne par la multiplicité des secteurs d’activités existants dans la région. En effet, non seulement cette industrie est importante dans l’économie bretonne, mais elle se présente aussi sous plusieurs formes.
			
	\section{La crise, impact et réponses}
		Bien que la Bretagne s’en sorte bien dans la crise de l’agro-alimentaire, cette dernière reste importante et pose de nombreux problèmes. Dans l’optique de savoir ce qui se passe autour de cette crise, nous commencerons par comparer ses impacts au niveau national et au niveau breton. Nous verrons ensuite quelles sont les actions menées par les pouvoirs publics. Enfin, nous parlerons des solutions envisagées pour l’avenir.
		
		\subsection{Des conséquences différentes entre la France et la Bretagne}
			\paragraph{}Avant de pouvoir parler de la réaction générale face à la crise, regardons pour commencer ses impacts. Afin de mieux nous rendre compte de la différence des conséquences en France et en Bretagne, nous procéderons tout d'abord à une analyse de l'impact aux différentes échelles, puis nous comparerons les résultats.

			\paragraph{}Dans un premier temps, intéressons-nous à ce que représente la crise au niveau national. En raison de l'importance que l'industrie agro-alimentaire a dans le pays, il paraît normal de penser que le secteur ait subi la crise de plein fouet, au moins sous certaines formes. La première des conséquences de la crise dans l'agro-alimentaire français est une conséquence de centralisation. En effet, les petites organisations résistent moins bien face à de tels évènements, poussant ainsi les plus grosses structures à effectuer des rachats ou à améliorer leur solidité. Par exemple, le groupe Intermarché a effectué le rachat des abattoirs Gad en Octobre 2014 pour les intégrer à sa filiale SVA Jean Rozé. Ces mêmes rachats ont pour but, entre autres, de diversifier leurs activités pour conforter une place qui tend à se fragiliser. En outre, avec les baisses des aides de la Politique Agricole Commune (pour l’exemple, l’Union Européenne a diminué de 24,5% les subventions européennes et de 45,4% les subventions nationales), les ressources financières du secteur se raréfient dangereusement, mettant en péril certaines branches de l’industrie. On peut citer, à titre d’exemple, la filière de la volaille, dont plusieurs représentants, à l’instar de Doux se retrouvent au bord de la fermeture. Cela provoque également une augmentation de l’importation, puisque le solde entre production et consommation diminue de plus en plus et de plus en plus vite (-22% de 1996 à 2000 puis -40% de 2000 à 2005).

			Qui plus est, la crise se manifeste aussi par une crise de confiance. Effectivement, outre la “vache folle de 1986 à 1996”, de multiples scandales se sont accumulés dans l’agro-alimentaire sur la dernière décennie : la grippe aviaire en 2006, les problèmes sur les produits importés la même année, le lait à la mélamine en 2008, l’Escherichia Coli dans les concombres en 2011, les steaks hachés contaminés en 2012, la fraude à la viande de cheval en 2013. À cause de cela, les modes de consommations changent. Plus précisément, les consommateurs se tournent vers les commerces de proximité, car les différents scandales sont majoritairement nés de l’import en masse des grands groupes industriels.
			
			D’un autre côté, des spécificités se présentent à l’échelle régionale. En Bretagne, l’agro-alimentaire est une industrie relativement importante pour l’économie, mais c’est surtout une force majeure pour la région, comme il a été dit précédemment. Ainsi, on voit que même si certaines organisations sont en difficulté, la présence de l’agro-alimentaire à différents niveaux permet de se passer du recours à des intermédiaires étrangers. Ce fait met donc en évidence l’existence d’une synergie entre les acteurs de la région.

			De ce fait, nous pouvons mettre en opposition deux situations : celle de la Bretagne et celle de la France. Ainsi, on voit que le secteur se retrouve dans une position difficile au niveau de la France, en raison de l’existence de marchés étrangers plus compétitifs et donc plus avantageux que les marchés locaux pour les distributeurs, mais aussi à cause du manque de confiance croissant des consommateurs. Mais l’état des choses en Bretagne est tout autre, puisque la région reste ancrée dans cette industrie, comme il a été précisé plus tôt, et que le soin que la région y porte lui offre un délai permettant de trouver des réponses plus efficaces et donc de faire plus aisément face à la crise.

			\paragraph{}Ainsi, nous pouvons sans trop de problèmes affirmer que la Bretagne est dans une meilleure situation que la France
			
		\subsection{Différentes actions à différentes échelles}
			\paragraph{}En réponse à cette crise, les pouvoirs publics ont mis en oeuvre de nombreuses actions, plus ou moins efficaces. Mais ces actions ne sont pas toutes venues du même échelon. En effet, certaines d'entre elles ont été menées au niveau national, d'autre à des niveaux plus locaux. Afin de mieux nous en rendre compte, nous commencerons par voir les actions du gouvernement. Ensuite nous verrons ce qui a été fait par la région et les départements.

			\paragraph{}Tout d’abord, le gouvernement a apporté des mesures dans le but de répondre aux problèmes de l’industrie agro-alimentaire. Pour commencer, le gouvernement a décidé en décembre 2013 de mettre en place le Plan Agricole et Agroalimentaire pour l’Avenir de la Bretagne au sein du Pacte d’Avenir. Cette action vise à apporter à la Bretagne des solutions à la fois pour pallier au déficit d’emplois de ces dernières années, mais aussi à soutenir les entreprises en difficulté. Pour ce faire, des subventions sont prévues pour les entreprises en difficulté, ce qui permet à la fois d’éviter les licenciements et d’aider les entreprises à se redresser. En outre, ce programme cherche à accorder plus d’importance au renouvellement des activités, par l’intermédiaire de l’innovation par exemple. De manière justement à mettre l’accent sur ce point, un soin tout particulier a été accordé aux pôles de compétitivité et aux instituts de recherche. Ainsi, l'une des initiatives prises est la mise en place de l’institut Carnot pour aider les entreprises à vérifier l’état des choses dans leur activité. L’importance donnée aux pôles de compétitivité vise quant à elle à rendre les entreprises capables d’améliorer et d’élargir leurs gammes de produits en fonction de ce qui est en vogue.

			Une autre des mesures prises l’est par la région. En effet, même si le conseil régional a participé à la création du Plan cité précédemment, des décisions se voulant à la fois plus précise et plus en accord avec la situation régionale ont été faites. La politique régionale, depuis le début de l’année 2014, est d’améliorer la qualité des produits proposés par l’industrie agro-alimentaire. Pour cela, l’accent est mis sur le respect de l’environnement et la production d’énergies propres. Nous pouvons constater que cela fait suite aux crises de l’agro-alimentaire qui ont provoqué la perte de confiance des consommateurs. Ainsi, en produisant mieux plutôt que plus, la région compte répondre aux attentes de la société actuelle en matière de consommation.

			Enfin, les entreprises elles-mêmes cherchent à prendre des mesures pour sauver leur activité. Nous pouvons, pour cela, citer une seconde fois le rachat des abattoirs Gad par Intermarché. Cette action se motive non seulement par des choix économiques dans le groupe, mais également par la volonté de fortifier la SVA Jean Rozé dans son activité d’exploitation de viande animale. Ainsi, en effectuant ce que l’on appelle l’intégration verticale, qui est pour une entreprise l’extension de ses activités le long de sa filière, le groupe se détache d’intermédiaires coûteux et présentant des risques pour sa pérennité. En outre, en étendant son activité, la filiale peut se pencher sur la question d’une modification de la gestion de son activité, et se moins se focaliser sur son cœur de métier tout en déplaçant des ressources sur la recherche et le développement de produits innovants.

			\paragraph{}En conclusion, des actions différentes ont été mises en place à plusieurs niveaux, mais il est important de noter que ces actions rejoignent majoritairement le même but : innover pour plaire au consommateur.

			
		\subsection{Quelles solutions pour l'avenir ?}
			Enfin, il est important de se rendre compte des solutions envisagées pour l’avenir de l’agro-alimentaire. 
			
\chapter{L'innovation, un remède contre la crise ...}
	
\section{Les différentes formes d'innovation}
	
		\subsection{L'innovation sur les modes de production}
		
\subsection{L'innovation sur l'organisation des entreprises}
		
\subsection{L'innovation par les nouvelles ressources}
		
\section{Comment fait-on pour innover ?}
		
\subsection{La recherche au sein des entreprises}
		
\subsection{Les acteurs à prendre en compte}
		
\subsection{La mise en place du produit}
		
\section{Les bénéfices des l'innovation}
	
\subsection{Une image revalorisée}
		
\subsection{De nouvelles opportunités}
		
\subsection{La création d'un nouveau marché}

\chapter{... mais qui ne réussit pas toujours}

\section{Une innovation pas vraiment optimale}

\subsection{Une vision trop portée sur le produit}
		
\subsection{Un processus d'innovation parfois trop rapide}
		
\subsection{L'agro-alimentaire, une industrie retardataire}
	
\section{L'innovation, inaccessible pour certains}
	
\subsection{Les contreparties à l'innovation}
		
\subsection{Le coût élevé}
		
\subsection{Une vision de spécialisation pour les petites structures}
		
\section{Les solutions actuelles}
	
\subsection{Les pôles de compétitivité}
		
\subsection{Les nouveaux modèles d’innovation}
		
\subsection{Les avancées en matière de recherche}

\chapter*{Conclusion}
\addcontentsline{toc}{chapter}{Conclusion}

\end{document}
